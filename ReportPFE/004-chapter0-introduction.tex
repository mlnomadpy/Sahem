% @Author: Taha Bouhsine


%%%%%%%%%%%%%%%%%%%%%%%%%%%%
% CHAPTER                  %
%%%%%%%%%%%%%%%%%%%%%%%%%%%%
\chapter*{Introduction}
\label{chap:general_intorduction}
\markboth{\MakeUppercase{Introduction}}{}%
\addcontentsline{toc}{chapter}{Introduction}%

One of the biggest topics discussed around the business community for the past few years is the idea of crowdfunding using an online platform. This is a great tool to use for building capital for a startup, funding growth for your company or development of services or products to further your business.
New crowdfunding platforms and websites are popping up regularly to meet the needs of the expanding market, with plenty of room to benefit from the advantages of this technology by creating a crowdfunding platform.

\section*{ Project Presentation }
\subsection*{Problem}
Our mission is to help bring creative projects to life.
privide a platform for people to create fundraisers.
funding platform for creative projects. Everything from films, games, and music to art, design, and technology. Kickstarter is full of ambitious, innovative, and imaginative ideas that are brought to life through the direct support of others.
backers pledge to projects to help them come to life and support a creative process. To thank their backers for their support, project creators offer unique rewards that speak to the spirit of what they're hoping to create.
A platform will allow for gatekeeping that will monitor and create symbiotic relations with other algorithms and information online. This can be done by using cloud-based solutions for better access.
we will be able to coordinate multiple campaigns easier.
we will be able to find people who are willing to invest with little equity involved.
our platform can help leap the hurdle of lack of experience in each field. Letting the barrier of entry be lessened for everyone who has a dream.
funding Model.
“Keep It All”
\subsection*{Objectifs}
build a crowdfunding platform whilist keeping in mind the following aspects.
\begin{enumerate}
    \item
          User Experience
          Making sure that our crowdfunding software solutions are easy to navigate for the end-user is a crucial part of the development process. If our customers are befuddled by how to navigate our application, then it is highly likely that they will leave and find another crowdfunding platform. Not only will we want a functional interface, but one that is eye-catching as well.
          One thing to take note of is the importance of giving them the terms and conditions within the first few steps, so they fully understand what to expect. This shows a dedication to transparency that many startups and entrepreneurs will appreciate.
    \item
          Account Management
          Our customers will want to know what is going on, and making it easy on them will help make our platform successful. That means setting up systems that make it very clear what’s going on with their project. We will want them to be able to access who has been investing, how much money they have, how far they are from their goal, and any other metric they need to run a successful campaign. This could also include reports for recharges, withdrawals all available via a simple to navigate dashboard.
    \item

          Report Generation
          As the platform owner, you need a way to benefit from your time spent on creating this site that will help so many people. So, ensuring you also have access to backed reports like rewards, investors, and such can help you help them, as they say. This means creating a dashboard that, just like for the actual campaign creators, is easy to navigate and gives you access to reports you can use to course-correct and upgrade the systems.
    \item
          Payment Gateway \& Marketing
          When starting a crowdfunding platform you will want to set it up with access to the right payment gateways. Each gateway has its features, and so doing some in-depth research into them will allow you to choose one or several that works for the largest number of potential customers.
\end{enumerate}



\subsection*{Document Organization}



\section*{Crowdfunding}
Crowdfunding is an increasingly popular alternative method of raising finance.
But what is crowdfunding? In this chapter we explain what crowdfunding is, how it works, the risks and rewards and Morocco's regulations on it.
\subsection*{Definition}
Crowdfunding is the practice of raising money from a large number of individuals for the purposes of financing a project, venture, business or cause. Traditionally, crowdfunding has been carried out via subscriptions, benefit events and door-to-door fundraising. However, today the term is typically associated with raising money through website platforms, which allows crowdfunding to reach a larger pool of potential funders.
\subsection*{History}


Crowdfunding has a long history with several roots. Books have been crowdfunded for centuries: authors and publishers would advertise book projects in praenumeration or subscription schemes. The book would be written and published if enough subscribers signaled their readiness to buy the book once it was out. The subscription business model is not exactly crowdfunding, since the actual flow of money only begins with the arrival of the product. The list of subscribers has, though, the power to create the necessary confidence among investors that is needed to risk the publication.

War bonds are theoretically a form of crowdfunding military conflicts. London's mercantile community saved the Bank of England in the 1730s when customers demanded their pounds to be converted into gold - they supported the currency until confidence in the pound was restored, thus crowdfunded their own money. A clearer case of modern crowdfunding is Auguste Comte's scheme to issue notes for the public support of his further work as a philosopher. The "Première Circulaire Annuelle adressée par l'auteur du Système de Philosophie Positive" was published on 14 March 1850, and several of these notes, blank and with sums have survived.
The cooperative movement of the 19th and 20th centuries is a broader precursor. It generated collective groups, such as community or interest-based groups, pooling subscribed funds to develop new concepts, products, and means of distribution and production, particularly in rural areas of Western Europe and North America. In 1885, when government sources failed to provide funding to build a monumental base for the Statue of Liberty, a newspaper-led campaign attracted small donations from 160,000 donors.

The late 19th century saw the creation of one of the world’s most recognisable landmarks
with the gift of the Statue of Liberty by the French to the US. While the French paid for the
construction and shipping of the statue it was down to the US to fund the base upon which it
would stand. With the statue ready to leave France, the Americans were still well short of the
\$300,0001
needed to build the base and erect the statue. Running short of time the American
Committee (responsible for raising the funds) teamed up with newspaper owner Joseph
Pulitzer to launch a campaign to invite citizens to donate even small amounts to help in the
funding of the pedestal, offering donors miniature replicas of the statue in return. This 19th
century crowdfunding campaign raised \$100,000 in just five months, contributed to one of
the most popular attractions in the world and illustrated the financing power of a large crowd
when tapped for funding.
While the American Committee were lucky to have Mr Pulitzer and his paper to publicise their
plea for donations, others wishing to access so many people would have had no such help.
This, however, has changed in recent years with the rise of social media and the new ease with
which communities can form and interact online.

Crowdfunding on the internet first gained popular and mainstream use in the arts and music communities.
The first noteworthy instance of online crowdfunding in the music industry was in 1997, when fans of the British rock band Marillion raised US\$60,000 in donations by means of an Internet campaign to underwrite an entire U.S. tour. The band subsequently used this method to fund their studio albums.
This built on the success of crowdfunding via magazines, such as the 1992 campaign by the Vegan Society that crowdfunded the production of the "Truth or Dairy" video documentary.
In the film industry, independent writer/director Mark Tapio Kines designed a website in 1997 for his then-unfinished first feature film Foreign Correspondents. By early 1999, he had raised more than US\$125,000 on the Internet from at least 25 fans, providing him with the funds to complete his film.
In 2002, the "Free Blender" campaign was an early software crowdfunding precursor.
The campaign aimed for open-sourcing the Blender 3D computer graphics software by collecting \$100,000 from the community while offering additional benefits for donating members.

The first company to engage in this business model was the U.S. website ArtistShare (2001).
As the model matured, more crowdfunding sites started to appear on the web such as Kiva (2005), IndieGoGo (2008), Kickstarter (2009), GoFundMe (2010), Microventures (2010), and YouCaring (2011).

The phenomenon of crowdfunding is older than the term "crowdfunding". According to wordspy.com, the earliest recorded use of the word was in August 2006.
\subsection*{ Before Crowdfunding, the Peer-To-Peer Lending Era }
Peer-to-peer lending, also abbreviated as P2P lending, is the practice of lending money to individuals or businesses through online services that match lenders with borrowers. Peer-to-peer lending companies often offer their services online, and attempt to operate with lower overhead and provide their services more cheaply than traditional financial institutions.[citation needed] As a result, lenders can earn higher returns compared to savings and investment products offered by banks, while borrowers can borrow money at lower interest rates,[1][2][3] even after the P2P lending company has taken a fee for providing the match-making platform and credit checking the borrower.

\subsection*{Functionality }
The platforms operate by allowing those seeking finance to make a pitch on the site outlining
how much money they need, what they need it for and what, if anything, you get in return for
contributing. Potential funders can then view pitches on the platform, interact with both those
looking for finance and other potential funders and then decide whether or not they want to
back the campaign. The majority of platforms operate the all–or–nothing model where, if the
target amount is not raised within a given timeframe, contributions are returned to funders and
no financing goes ahead.
\subsection*{ Crowdfunding Types }
Crowdfunding facilitates the raising of capital for a variety of purposes, using numerous
variations of the model. Below is a typology of how the operators in the market can potentially
be segregated. The majority of platforms can be categorised under these four types, but there
are several variations, such as hybrid models and those platforms that define themselves in a
sectoral vertical rather than by the type of finance they provide.
\begin{enumerate}
    \item Donation Model:
          The donation model of crowdfunding is a means for charities, or those who
          raise money for social or charitable projects, to gather a community online and to enable
          them to donate to a project. While most established charities facilitate this through their
          own website, crowdfunding is popular for small organisations and people raising money for
          personal or specific charitable causes. Popular sites include Crowdrise and Causes.
    \item Lending Model:
          Crowdfunded lending is largely an evolution of the peer–to–peer model of
          lending, pioneered by firms such as Lendingclub and Zopa. Projects or businesses seeking
          debt apply through the platform uploading their pitch, with members of the crowd taking
          small chunks of the overall loan. Some platforms focused on social causes offer interest–free
          loans such as micro–lending site Kiva. Others operate more as an investment, where interest
          rates are decided either by those seeking the loan or using a market for loan parts, such as that
          used by UK platform FundingCircle.
    \item Reward Model:
          The most popular form of crowdfunding to date has been the reward model
          which has grown significantly in the funding of creative, social and entrepreneurial projects.
          The model allows people to contribute to projects and receive non–financial rewards in return,
          usually operating a tiered system where the more you donate the better the reward you
          receive. The model often closely resembles philanthropy with the donation far exceeding the
          monetary value of the reward or the reward costing the fundraiser little, such as experience or
          recognition–related rewards. For some projects the model is similar to a presale agreement.
          In these cases entrepreneurs or artists crowdfund the production cost of their record, movie,
          game or product and allow the donors to be the first recipients once the production is
          complete. Popular platforms operating the reward model are Kickstarter and Indiegogo in the
          US and Peoplefund.it in the UK.
    \item Investment Model:
          The final type is the application of crowdfunding to investing for equity, or
          profit/revenue sharing in businesses or projects. This form of the model has been the slowest
          to grow due to regulatory restrictions that relate to this type of activity. Some European
          platforms have been pioneers of the equity crowdfunding model, allowing anyone to take a
          small stake in an unlisted or private business through crowdfunding. The most popular sites in
          offering this model are CrowdCube in the UK and Symbid in the Netherlands. Others such as
          Quirky offer a revenue or profit–sharing model allowing you to capitalise on the success of the
          projects you back.
\end{enumerate}


\subsection*{ How big is crowdfunding? }
In 2015, crowdfunding raised \$34 Billion worldwide (Source: Massolutions) and the total
had doubled every year for the previous four years! Many crowdfunding campaigns are
for quite small sums but some are very large. The model of crowdfunding which raised
the most money is the lending model. In the UK crowdfunding in 2015 totalled £1,112
million and this figure is expected to continue to rise.


\subsection*{ Why is crowdfunding growing? }
Making a public call to fundraise from the crowd is not a new idea but crowdfunding
as we now understand it has grown very quickly for a number of reasons since its
emergence in the 1990s. These include:
\begin{enumerate}
    \item
          Technological developments : The emergence of wide and low-cost access to the
          internet and communication tools like social media mean it is easier and cheaper for us
          to reach out to much more widely dispersed and larger groups of people.
    \item
          Societal changes : These technical changes have also empowered us to take on
          new activities which were once controlled by gatekeepers. We can see this in the way
          people write and publish books, publish music, writing blogs, and the general sharing
          of our lives online. Crowdfunding is just the financial manifestation of this sense of
          empowerment and the ability to take “ownership” of a process. At the same time, we
          are also increasingly comfortable with transacting financially online be it shopping
          and e-commerce or checking our bank balances. This confidence to use money online
          is essential for the growth of crowdfunding.
    \item
          Economic factors : The other key factor in the extraordinary growth of crowdfunding is
          that after the financial crisis of 2008 access to funding has been more problematic and
          so people are exploring alternatives to the traditional sources. At the same time interest
          rates have fallen to historically low levels and so “retail investors” are looking for better
          places to put their investments and some crowdfunding campaigns seem to offer better
          returns than would be available on the high street.
\end{enumerate}

\subsection*{Principe }
The other important variation in crowdfunding is the distinction between what are
known as the “Keep It All” and “All Or Nothing” models.\\
In a “Keep It All” campaign, you keep everything you raise regardless of whether you
reach your target or not.\\
In an “All Or Nothing” campaign you only get to keep what you raise if you succeed in
reaching your target.\\
Crowdfunding can be undertaken by both individuals and organisations.


%%%%% 
\subsection*{ Downsides of Crowdfunding}
\begin{enumerate}
    \item It is not easy – to be a success takes a lot of time and effort to carry out continued
          promotion of your campaign.
    \item Many campaigns are unsuccessful and the successful ones take work, preparation
          and effort to make them happen – there is no guarantee of success but work,
          preparation and effort can increase the chances of success.
    \item It is also a very public process and so you must be prepared to be open and honest in
          a public arena and expect brickbats and bouquets in equal measure – it is important
          to accept that there may be public scrutiny of your project and prepare for this.
    \item Reputation considerations:
          there are a number of aspects to this, including your appearance on the platform in the first place. Might customers and others view it negatively? There are also significant expectations from investors if your crowdfunding campaign is successful. If it meets delays or runs into problems, the reputation of your company might become damaged. Finally, by putting your idea onto a crowdfunding platform before you’re ready to bring it to the wider market (i.e. before it’s fully developed and tested), you are leaving your company open to public criticism of your plans, idea, or business.
\end{enumerate}
%%%%%%%


\subsection*{ Benefits of Crowdfunding }
\begin{enumerate}
    \item It is a very accessible process which is open to all and can be carried out on your terms
    \item you decide on the amount you want to raise and the timescale to raise the funds.
    \item You are in control – the promotion and selling of the project is the responsibility of
          your group.
    \item It can bring much more than money – it can attract new people and support (nonfinancial) to your group.
    \item What money it does bring can be very different from traditional investment – funds
          raised through crowdfunding are unrestricted and can be used for all elements of
          your project.
    \item It can be quick – as you are in control you are able to work quickly to start raising funds.
    \item You will also produce a very valuable asset for you or your organisation as you run a
          campaign and that “crowd asset” can be a very useful and enduring resource –
          people who support your project are also likely to support your group.
    \item It’s a quick and easy way to get a lot of exposure for your brand and for your new idea or initiative
    \item You can engage directly with investors who could also become your customers
    \item There is a low barrier to entry – often much lower than with other forms of investment
    \item It’s an easy way to get feedback on your idea
\end{enumerate}


\subsection*{ What makes crowdfunding different from other funding? }
Crowdfunding reaches widely by using technology and reduces the size of funding each
individual contributor has to come up with. This means that more people can take part.
Making it easy for a wider group of people to support a business or project introduces
a wider range of motivations for people to back a campaign. This means that there is a
range of reasons why people might support you, and not simply for a financial return.
The idea of many small contributions making a difference (as opposed to a small
number of large contributions) is a concept underpinning lots of online activities which
have disrupted traditional industries and it is called “The Long Tail”.
Because the process of crowdfunding is a very public one and involves many people it
can, and does, bring many more advantages than simply money. It can be a powerful
campaigning tool, it can build networks, validate an idea, build awareness and many
other things, all of which can be very valuable and useful. A good crowdfunding
campaign will recognise and target these additionalities.
For some entrepreneurial projects, it has the advantage of accelerating the process
of setting up a business by allowing the entrepreneur to run in parallel a series of
processes like market research, publicity and marketing and fundraising, which have
often traditionally been seen as sequential.



\subsection*{ Motivation And Rewards }



\subsection*{ Crowdfunding Actors }
in crowdfunding there is two main actors, the project owner that created the fundraiser, and the actor that provide the funds.
\begin{enumerate}
    \item Creator
    \item Funder general public
\end{enumerate}

\subsection*{ Existent Platforms }
Non-Profit:
\begin{enumerate}
    \item
          Fundly: This is probably the best known of the non-profit crowdfunding sites. This platform charges a fee for all campaigns and a small transaction fee for processing. It offers a customizable donation page, with an integrated mobile experience. You can use multiple media formats to promote your cause as well as blogging.
    \item
          Salsa: This p2p platform offers branded marketing materials for every supporter and customizes messages as well. This is a fully integrated system with Salsa CRM. So, if you already use that, software data is transferred easily.
    \item
          Soapbox Engage: This is a social change platform that offers not only the ability to create cause for donation but allows you to create forms, petitions, and events to further your cause.
    \item
          Bonfire: is a crowdfunding platform that allows you to create custom t-shirts to raise money for your projects. You create the t-shirt and then build a custom web page to market and use as a landing spot to drive your social media marketing campaigns.
\end{enumerate}
Business:
\begin{enumerate}
    \item
          Kickstarter: This is a crowdfunding platform that allows you to market potential products or businesses for investors to donate to. You create a custom page for your book, game, or any other creative project and then set a goal and start building funds. Each project will set up designated donations that have rewards attached to them.
    \item
          Indigogo: This is a great sight for startups and creatives that uses similar methods to Kickstarter. You set up a custom page and goals and market your campaign. They have integrated systems to help with the fulfillment of delivery, mobile management, perk options, etc.
    \item
          Seedrs: This crowdfunding platform uses equity investing to raise funds for small businesses and startups. It has three options to invest whether equity, funds or convertible donations and allows other investors a discount in the future.
    \item
          Crowdcube: This platform offers you the chance to invest in business and causes via an easy-to-use website. Creating campaigns via your own customizable page, you will have access to analytics and social media campaigns to help promote your project.
\end{enumerate}




\subsection*{ Crowdfunding In Morocco }
Expected since April 2018, the examination of the so-called “collaborative financing” bill is finally starting in Morocco. Technically, the concept of crowdfunding (“crowd financing”) is to bring together two needs through a platform on the internet: that of the entrepreneur/idea carrier who lacks liquidity and hopes to find financing, and that of a saver who wants to invest a certain amount of money, even modest, in a project with detailed outlines. These are generally small businesses, but this law will also enable associations to raise funds to finance their projects.

Under the banner of crowdfunding, the law admits three types of financing operations, each time via an electronic platform edited and managed by a collaborative financing company: the financing of projects in the form of a loan (crowdlending, for which Bank Al-Maghrib will control the interest rate or the maximum duration of the loan), a grant (crowdfunding: the donor will have to obtain an authorization if the amount exceeds MAD 500,000), or capital (crowd equity).

In the latter case, for management companies that want to propose capital investment, it will be necessary to obtain a visa from the Moroccan Capital Market Authority (AMMC). The platforms must have a minimum capital of \$31,000 (MAD 300,000) and the collaborative finance companies that will manage them must have a risk prevention and reduction policy.

% crowdfunding in morocco 
he majority of respondents 96\% do not even know the word crowdfunding.
We also noticed that there is only one person who has already contributed to or supported a project through crowdfunding and this has two causes: the first, as we have already mentioned, is that the majority do not know this method of financing and the second is, of course, the absence of a regulation framing this system. We also found that more than half of the respondents are willing to support a project by crowdfunding for a percentage of 50.5\%, for those who answered no, it is mainly the lack of information and money that is the reason why they are not ready to support a project via crowdfunding. Regarding the type of project most respondents are ready to support a charitable project, social or solidarity with a percentage of 81.6\%, followed by support from a local company with 75.7\% and then the development of a business with a high growth potential with a percentage of 59.2\%, this shows that Moroccans have the culture of giving and helping, even if the amounts remain low because the majority are ready to support a project for an amount that does not exceed 500 DH. Most respondents plan to have their own projects with a percentage of 62\%, while 5\% answered I do not know against 33\% do not want to have their projects. However, according to the distribution by labor force, we find that this majority consists mainly of the unemployed and students, while more than half of the workers (58.7\%) do not want to have their own projects. For those who want to have  their own projects, it is mainly the lack of funding (84.7\%), and the lack of idea and the  risk of failure  that  prevents  them,  so  crowdfunding  can  fill  this  gap  is  becoming  an  alternative  to  financing projects  or  even  a complement.  For  those  who  do  not  want  to  have  their  projects  they  are  mainly  workers  and  are  satisfied  by  their  work  or students who want to finish their studies first. Finally we found that the majority who want to have their own project and who have financing problems for a percentage of 55.5\% are interested in crowdfunding to launch or develop their projects, however the lack of information on this system is the first cause for those who are not interested in this new mode of financing, which is normal. So, we can answer the hypothesis that we mentioned in the introduction: What  is  the  future  of  crowdfunding  in  Morocco?  Would  it  be  a  realistic  alternative  to  overcome  the  lack  of  funding  for startups, very small company - small and medium company in Morocco? What are the stakes and obstacles that hinder this type of financing? According  to  the  respondents'  answers  and  according  to  the  second  part  where  we  have  focused  on  crowdfunding  in Morocco,  we  can  say  that  this  new  financing  mode  can  find  its  place  in  Morocco  and  its  future  is  certain,  and  can  be  an indispensable instrument for the financing of start-up companies and a complement to the traditional financing circuit for very small, medium and small companies, although there are several problems to be solved such as the ignorance or the confusion of the people toward this concept or the absence of regulations governing crowdfunding in Morocco.

According  to  the  respondents's answers  and  according  to  the  second  part  where  we  have  focused  on  crowdfunding  in Morocco,  we  can  say  that  this  new  financing  mode  can  find  its  place  in  Morocco  and  its  future  is  certain,  and  can  be  an indispensable instrument for the financing of start-up companies and a complement to the traditional financing circuit for very small, medium and small companies, although there are several problems to be solved such as the ignorance or the confusion of the people toward this concept or the absence of regulations governing crowdfunding in Morocco.

\section*{ Project management }
\subsection*{ Project Development Proccess }
\subsection*{ Monitoring and planning }